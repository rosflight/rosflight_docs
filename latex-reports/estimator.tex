%%%%%%%%%%%%%%%%%%%%%%%%%%%%%%%%%%%%%%%%%
% Short Sectioned Assignment
% LaTeX Template
% Version 1.0 (5/5/12)
%
% This template has been downloaded from:
% http://www.LaTeXTemplates.com
%
% Original author:
% Frits Wenneker (http://www.howtotex.com)
%
% License:
% CC BY-NC-SA 3.0 (http://creativecommons.org/licenses/by-nc-sa/3.0/)
%
%%%%%%%%%%%%%%%%%%%%%%%%%%%%%%%%%%%%%%%%%

%----------------------------------------------------------------------------------------
%	PACKAGES AND OTHER DOCUMENT CONFIGURATIONS
%----------------------------------------------------------------------------------------

\documentclass[paper=a4, fontsize=11pt]{scrartcl} % A4 paper and 11pt font size

\usepackage[T1]{fontenc} % Use 8-bit encoding that has 256 glyphs
% \usepackage{fourier} % Use the Adobe Utopia font for the document - comment this line to return to the LaTeX default
\usepackage[english]{babel} % English language/hyphenation
\usepackage{amsmath,amsfonts,amsthm} % Math packages

\usepackage{lipsum} % Used for inserting dummy 'Lorem ipsum' text into the template

\usepackage{sectsty} % Allows customizing section commands
\allsectionsfont{\centering \normalfont\scshape} % Make all sections centered, the default font and small caps

\usepackage{fancyhdr} % Custom headers and footers
\usepackage{bm}
\usepackage{graphicx}
\usepackage{upgreek}
\pagestyle{fancyplain} % Makes all pages in the document conform to the custom headers and footers
\fancyhead{} % No page header - if you want one, create it in the same way as the footers below
\fancyfoot[L]{} % Empty left footer
\fancyfoot[C]{} % Empty center footer
\fancyfoot[R]{\thepage} % Page numbering for right footer
\renewcommand{\headrulewidth}{0pt} % Remove header underlines
\renewcommand{\footrulewidth}{0pt} % Remove footer underlines
\setlength{\headheight}{13.6pt} % Customize the height of the header

\numberwithin{equation}{section} % Number equations within sections (i.e. 1.1, 1.2, 2.1, 2.2 instead of 1, 2, 3, 4)
\numberwithin{figure}{section} % Number figures within sections (i.e. 1.1, 1.2, 2.1, 2.2 instead of 1, 2, 3, 4)
\numberwithin{table}{section} % Number tables within sections (i.e. 1.1, 1.2, 2.1, 2.2 instead of 1, 2, 3, 4)

%----------------------------------------------------------------------------------------
%	TITLE SECTION
%----------------------------------------------------------------------------------------

\newcommand{\horrule}[1]{\rule{\linewidth}{#1}} % Create horizontal rule command with 1 argument of height

\title{
\normalfont \normalsize
\textsc{MAGICC Lab - Brigham Young University} \\ [25pt] % Your university, school and/or department name(s)
\horrule{0.5pt} \\[0.4cm] % Thin top horizontal rule
\huge Estimator \\ % The assignment title
\horrule{2pt} \\[0.5cm] % Thick bottom horizontal rule
}

\author{James Jackson} % Your name

\date{\normalsize\today} % Today's date or a custom date

\begin{document}

\maketitle % Print the title

%----------------------------------------------------------------------------------------
%	PROBLEM 1
%----------------------------------------------------------------------------------------

\section{Introduction}

The estimator is used to calculate an estimate of the attitude and angular velocity of the multirotor.  It is assumed that the flight controller is mounted rigidly to the body of the aircraft (perhaps with dampening material to remove vibrations from the motors), such that measurements of the onboard IMU are consistent with the motion of the aircraft.

Due to the limited computational power on the embedded processor, and to calculate attitude estimates at speeds up to 1000Hz, a simple complementary filter is used, rather than an extended Kalman filter.  In practice, this method works extremely well, and it used widely throughout commercially available autopilots.  There are a variety of complementary filters, but the general theory is the same.  A complementary filter is a method to fuse the measurements from a gyroscope, accelerometer and sometimes magnetometer to produce an estimate of the attitude of the MAV.

\section{Sensors}
There are a variety of sensors available on most flight control units, but we will limit our discussion to accelerometers, rate gyroscopes and magnetometers.  Each of these is important in estimating the attitude of an MAV.

\subsection{Accelerometers}

Accelerometers measure all the forces acting on the aircraft.  We can assume that these forces consist primarily of forces from the propellers, but it often also consists of vibrations from unbalanced motors and propellers, wind, ground effect, and a variety of other sources. One thing accelerometers do \textit{not} measure is acceleration due to gravity.  As a result, if you hold an accelerometer still and look at it's measurements, you'll notice it measures the forces required to hold it in place.  For example, if you set an accelerometer on a table, it will measure the normal force the table exerts on the accelerometer.  If we assume that the accelerometer is at rest (albeit a rather significant assumption for a MAV), then we can assume that the forces acting on the accelerometer must be directly opposite gravity.  Knowing then the direction of gravity with respect to the body-fixed axes of the MAV allows us to back out the MAV's roll and pitch angles.  (see Figure~\ref{fig:accelerometer_measure}.)

\begin{figure}[h]
\centering
% \includegraphics[width=0.5\textwidth]{fig/accelerometer_measurement.png}
\caption{Illustration of acceleromemeter measurement if MAV is at steady state.  Because the accelerometer does not measure gravity, if we assume that the MAV is not accelerating, then the only accelerations measured are those which counteract gravity.  The angle $\phi$ can be therefore calculated using the vector $y_{acc}$}
\label{fig:accelerometer_measure}

\end{figure}

As you might expect, a MAV is not always at rest, so accelerometer measurements tend to bounce around as the MAV accelerates to move around.  In addition, there are often also a lot of high-frequency external forces (such as vibrations induced by imbalanced propellers and motors) on a MAV in flight which adds noise to the accelerometer measurement.  We can assume, however, that these accelerations are short-lived, and the accelerometer measurement remains stable over time.

Modern MEMS accelerometers suffer from a significant degree of temperature-dependent bias.  This can be compensated for by recording the value of the stationary bias for all axes of the accelerometer vs temperature, and performing a least-squares approximation.  In ROSflight, this approximation is first-order, and takes place on the onboard computer to take advantage of the superior computing power and ability to handle large amounts of data.  Biases and compensation coefficients are then sent and applied on the flight controller.

\subsection{Rate Gyrocopes}

Rate Gyroscopes, or gyros, on the other hand, directly measure the angular velocity of the IMU.  Gyros are not susceptible to external forces like accelerometers, and instead measure angular velocity directly, but integrating gyroscopes over time to calculate attitude would lead to large amounts of drift.  Common MEMS rate gyros also often have some non-zero bias which must be corrected in order for their use on MAVs. This bias wanders in a stochastic manner, and cannot be corrected through temperature compensation.  It can, however, be estimated using certain filters.

\subsection{Magnetometers}

Magnetometers are often included on many flight control units, and in theory, act like a compass by measuring earth's magnetic field.  Using both the magnetometer and the accelerometer allows you to solve for both the attitude and heading of the vehicle.  Without a magnetometer, the MAV's attitude roll and pitch angles can be estimated, but the heading angle is unobservable.

However, because most MAVs now use electric motors, the magnetic field created by these motors, and high-current electricity flowing throughout the MAV, the magnetic field is subject to a very large amount of noise during flight.  As a result, magnetometers are not very reliable in practice, and should be supplanted instead with GPS.  In this derivation, we will assume that we are not using a magnetometer.

The magnetometer however, is used very similarly to the accelerometer. In the abscence of other magnetic noise, the vector returned by the magnetometer measurement can be compared directly with the known inertial magnetic field.  If you desire to incorporate the magnetometer into your estimator, you can simply copy the accelerometer update equations and instead of using the gravity vector as a reference, use the vector describing earth's magnetic field at your location.

\section{Complementary Filtering}
The idea behind complementary filtering is to try to get the ``best of both worlds'' of gyros and accelerometers.  Gyros are very accurate in short spaces of time, but they are subject to drift.  Accelerometers don't drift in the long scheme of things, but they are noisy as the MAV moves about.  So, to solve these problems, the complementary filter primarily propagates states using gyroscope measurements, but then corrects drift with the accelerometer.  In a general sense, it is like taking a high-pass filtered version of gyroscope measurements, and a low-pass filtered version of accelerometers, and fusing the two together in a manner that results in an estimate that is stable over time, but also able to handle quick transient motions.

\section{Attitude Representation}
There are a number of ways to represent the attitude of a MAV.  Most commonly, attitude is represented in terms of the euler angles yaw, pitch and roll, but it can also be represented in other ways, such as rotation matrices, and quaternions.

\subsection{Euler Angles}
Euler angles represent rotations about three different axes, usually, the z, y, and x axes in that order.  This method is often the most easy for users to understand and interpret, but it is by far the least computationally efficient.  To propagate euler angles, the following kinematics are employed:
\begin{equation}
	\begin{bmatrix}
		\dot{\phi} \\
		\dot{\theta} \\
		\dot{\psi} \\
	\end{bmatrix}
	=
	\begin{bmatrix}
		1 & \sin(\phi) \tan(\theta) & \cos(\phi)\tan(\theta) \\
		0 & \cos(\phi)              & -\sin(\phi) \\
		0 & \frac{\sin(\phi)}{\cos(\theta)} & \frac{\cos(\phi)}{\cos(\theta)}  \\
	\end{bmatrix}
	\begin{bmatrix}
		p \\
		q \\
		r \\
	\end{bmatrix}
\end{equation}

Note the large number of trigonometric functions associated with this propagation.  In a complementary filter, this will be evaluated at every measurement, and the non-linear coupling between $\bm{\omega}$  and the attitude becomes very expensive, particularly on embedded processors.

Another shortcoming of euler angles is known as ``gimbal lock''.  Gimbal lock occurs at the ``singularity'' of the euler angle representation, or pointing directly up or down.  The problem occurs because there is more than one way to represent this particular rotation.  There are some steps one can take to handle these issues, but it is a fundamental problem associated with using euler angles, and motivates the other attitude representations.

\subsection{Rotation Matrix}

Rotation matrices, are often used in attitude estimation, because they do not suffer from gimbal lock, are quickly converted to and from euler angles, and because of their simple kinematics.

\begin{equation}
	\dot{R} = \lfloor\bm{\omega}\rfloor R
\end{equation}

where $\lfloor \bm{\omega} \rfloor$ is the skew-symmetric matrix of $\bm{\omega}$, and is related to calculating the cross product.

\begin{equation}
	\lfloor\bm{\omega}\rfloor =
	\begin{bmatrix}
		0 & -r & q \\
		r & 0 & -p \\
		-q & p & 0
	\end{bmatrix}
\end{equation}
  This propagation step is linear with respect to the angular rates, which simplifies calculation significantly.

A rotation matrix from the inertial frame to body frame can be constructed from euler angles via the following formula:
\newcommand{\ct}{c\theta}
\newcommand{\cp}{c\phi}
\newcommand{\cs}{c\psi}
\newcommand{\st}{s\theta}
\newcommand{\sphi}{s\phi}
\newcommand{\spsi}{s\psi}
\begin{equation}
	    R = \begin{bmatrix}
	    	\ct\cs & \ct\spsi & -\st \\
        \sphi\st\cs-\cp\spsi & \sphi\st\spsi+\cp\cs & \sphi\ct \\
        \cp\st\cs+\sphi\spsi & \cp\st\spsi-\sphi\cs & \sphi\st \\
        \end{bmatrix}
\end{equation}

and converting back to euler angles is done via the following;
\begin{equation}
  \begin{bmatrix}
    \phi \\
    \theta \\
    \psi \\
  \end{bmatrix} =
	\begin{bmatrix}
	 \textrm{atan2}\left(R_{32}, R_{33}\right) \\
   \textrm{atan2}\left(-R_{31}, \sqrt{R_{21}^2 + R_{33}^2}\right) \\
   \textrm{atan2}\left(R_{21}, R_{11}\right) \\
  \end{bmatrix}
\end{equation}

\subsection{Quaternions}

Quaternions are a number system which extends complex numbers.  They have four elements, commonly known as $w$, $x$, $y$, and $z$.  The last three elements can be though of as describing an axis, $\beta$ about which a rotation occurred, while the first element, $w$ can be though of as describing the amount of rotation $\alpha$ about that axis. (see eq~\ref{eq:euler_to_axis_angle}). While this may seem straight-forward, quaternions are normalized so that they can form a group.  (That is, a quaternion multiplied by a quaternion is a quaternion), so they end up being really difficult for a human being to interpret just by looking at the values.  However, they provide some amazing computational efficiencies, most of which comes from the following special mathematics associated with quaternions.

First, just the definition of a quaternion:

\begin{equation}
	\bm{q} = \begin{bmatrix}
				q_w \\
				q_x \\
				q_y \\
				q_z
			 \end{bmatrix}
		   = \begin{bmatrix}
				\cos(\alpha/2) \\
				\sin(\alpha/2)\cos(\beta_x) \\
				\sin(\alpha/2)\cos(\beta_y) \\
				\sin(\alpha/2)\cos(\beta_y)
			 \end{bmatrix}
			 = \begin{bmatrix}
			   s \\
			   \bm{v}_x \\
			   \bm{v}_y \\
			   \bm{v}_z
			 \end{bmatrix}
	\label{eq:euler_to_axis_angle}
\end{equation}

and second, some formulas to convert to and from euler angles.


\begin{equation}
	\bm{q} =
	  \begin{bmatrix}
	  	\cos(\theta/2)\cos(\theta/2)\cos(\psi/2) + \sin(\phi/2)\sin(\theta/2)\sin(\psi/2) \\
	  	\sin(\theta/2)\cos(\theta/2)\cos(\psi/2) - \cos(\phi/2)\sin(\theta/2)\sin(\psi/2) \\
	  	\cos(\theta/2)\sin(\theta/2)\cos(\psi/2) + \sin(\phi/2)\cos(\theta/2)\sin(\psi/2) \\
	  	\cos(\theta/2)\cos(\theta/2)\sin(\psi/2) - \sin(\phi/2)\sin(\theta/2)\cos(\psi/2) \\
	  \end{bmatrix}
\end{equation}

\begin{equation}
	\begin{bmatrix}
		\phi \\
		\theta \\
		\psi \\
	\end{bmatrix}
	=
	\begin{bmatrix}
		\textrm{atan2}\left( 2\left(q_wq_x + q_yq_z\right),1-2\left(q_x^2+q_y^2\right) \right) \\
		\textrm{sin}^{-1}\left( 2 \left( q_wq_y - q_zq_x \right)  \right) \\
		\textrm{atan2}\left( 2 \left( q_wq_z + q_xq_y \right), 1 - 2 \left( q_y^2 q_z^2 \right)  \right)
	\end{bmatrix}
	\label{eq:euler_from_quat}
\end{equation}

Quaternions are ``closed'' under the following operation, termed quaternion multiplication.

\begin{equation}
	\bm{q_1} \otimes \bm{q_2} = \begin{bmatrix}
									s_1s_2 - \bm{v_1}^\top\bm{v_2} \\
									s_1\bm{v_2} + s_2\bm{v_1} + \bm{v_1} \times \bm{v_2}
								\end{bmatrix}
\end{equation}

To take the ``inverse'' of a quaternion is simply to multiply $s$ or $\bm{v}$ by $-1$

\begin{equation}
	\bm{q}^{-1} = \begin{bmatrix}
				-q_w \\
				q_x \\
				q_y \\
				q_z
			 \end{bmatrix}
			 	= \begin{bmatrix}
				q_w \\
				-q_x \\
				-q_y \\
				-q_z
			 \end{bmatrix}
\end{equation}

and to find the difference between two quaternions, just quaternion multiply the inverse of one quaternion with the other.

\begin{equation}
	\tilde{\bm{q}} = \hat{\bm{q}}^{-1} \otimes \bm{q}
\end{equation}

However, the most important aspect of quaternions is the way we propagate dynamics.

\begin{equation}
	\dot{\bm{q}} = \frac{1}{2} \bm{q} \otimes \bm{q_{\omega}}
\end{equation}

where $\bm{q_{\omega}}$ is the pure quaternion created from angular rates.

\begin{equation}
	\bm{q_{\omega}} = \textrm{p}(\bm{\omega}) =
			  \begin{bmatrix}
				0 \\
				p \\
				q \\
				r
			 \end{bmatrix}
\end{equation}

What this means is that, like rotation matrices, quaternion dynamics are \textit{linear} with respect to angular rates, as opposed to euler angles, which are non-linear, and they take less computation than rotation matrices because they have fewer terms.  Casey et al.~\cite{Casey2013} performed a study comparing all three of the above representations, and found that complementary filters using an euler angle representation took \textbf{12 times} longer to compute on average than a quaternion-based filter.  Quaternions were also about 20\% more efficient when compared with rotation matrices.  For these reasons, ROSflight uses quaternions in its filter.

\section{Derivation}

ROSflight implements the quaternion-based passive ``Mahony'' filter as described in~\cite{Mahony2007}.  In particular, we implement equation 47 from that paper, which also estimates gyroscope biases.  A Lyuapanov stability analysis is performed in that paper, in which it is shown that all states and biases, except heading, are globally asymptotically stable given an accelerometer measurement and gyroscope.  The above reference also describes how a magnetometer can be integrated in a similar method to the accelerometer.  That portion of the filter is ommitted here due to the unreliable nature of magnetometers onboard modern small UAS.

\subsection{Passive Complementary Filter}
The original filter propagates per the following dynamics:

\newcommand{\wacc}{\bm{\omega}_{acc}}
\begin{equation}
	\begin{aligned}
	\bm{\dot{\hat{q}}} &= \frac{1}{2} \bm{\hat{q}} \otimes \bm{q}_{\omega}\\
	\bm{\dot{\hat{b}}} &= -2k_I\wacc
	\end{aligned}
	\label{eq:traditional_prop}
\end{equation}


where $\bm{q}_{\omega}$ is the pure quaternion formed from a composite $\bm{\omega}_{comp}$ which consists of the estimated angular rates, $\bar{\bm{\omega}}$, the estimated gyroscope biases, $\hat{\bm{b}}$, and an adjustment term, calculated from accelerometer measurements, $\bm{\omega}_{acc}$. $k_p$ and $k_I$ are constant gains used in determining the dynamics of the filter.
\begin{equation}
\begin{aligned}
	\bm{q}_{\omega} &= \textrm{p}\left(\bar{\bm{\omega}} - \hat{\bm{b}} + k_P\wacc\right) \\
	&= \textrm{p}\left( \bm{\omega}_{comp} \right)
	\label{eq:q_omega}
\end{aligned}
\end{equation}

\newcommand{\avec}{\bm{a}}
\newcommand{\gvec}{\bm{g}}
\newcommand{\qmeas}{\bm{q}_{acc}}
\newcommand{\gamvec}{\bm{\gamma}}
\newcommand{\norm}[1]{\left\lVert#1\right\rVert}
\newcommand{\wbar}{\bm{\bar{\omega}}}
\newcommand{\w}{\bm{\omega}}
\newcommand{\qhat}{\hat{\bm{q}}}

$\wacc$ can be described as the error in the attitude as predicted by the accelerometer.  To calculate $\wacc$ the quaternion describing the rotation between the accelerometer estimate and the inertial frame, $\qmeas$, is first calculated
\begin{equation}
\begin{aligned}
	\avec &=
		\begin{bmatrix}
			a_x \\
			a_y \\
			a_z \\
			\end{bmatrix},
	\qquad
	\gvec &=
	  \begin{bmatrix}
			0 \\
			0 \\
			1 \\
		\end{bmatrix}, \qquad
	\gamvec &= \frac{\avec+\gvec}{\norm{\avec+\gvec}}, \qquad
	\qmeas &=
	  \begin{bmatrix}
	    \avec^\top \gamvec \\
	    \avec \times \gamvec
	  \end{bmatrix}
\end{aligned}
\end{equation}

\newcommand{\qtilde}{\tilde{\bm{q}}}
\newcommand{\vtilde}{\tilde{\bm{v}}}
Next, the quaternion error between the estimate $\qhat$ and the accelerometer measure $\qmeas$ is calculated.
\begin{equation}
	\qtilde = \qmeas^{-1} \otimes \qhat =
		\begin{bmatrix}
			\tilde{s}\\
			\vtilde
		\end{bmatrix}
\end{equation}

\newcommand{\wmeas}{\bm{\omega}_{acc}}
Finally, $\qmeas$ is converted back into a 3-vector per the method described in eq. 47a of~\cite{Mahony2007}.
\begin{equation}
	\wmeas = 2\tilde{s}\vtilde
\end{equation}

Although the formulation in Eq.~\ref{eq:traditional_prop} is more concise, in implementation, $\bm{q}_{\omega}$ is never actually formed, instead the composite $\bm{\omega}_{comp}$ vector of Eq.~\ref{eq:q_omega} is directly substituted into equation~\ref{eq:mat_exponential}.  Gyroscope bias propagation is performed using standard euler integration, because they are assumed to be nearly constant.

\subsection{Modifications to Original Passive Filter}
There have been a few modifications to the passive filter described in~\cite{Mahony2007}, consisting primarily of contributions from~\cite{Casey2013}.  Firstly, rather than simply taking gyroscope measurements directly as an estimate of $\bm{\omega}$, a quadratic polynomial is used to approximate the true angular rate from gyroscope measurements to reduce error.  In~\cite{Casey2013}, this process was shown to reduce RMS error by more than 1,000 times.  There are additional steps associated with performing this calculation, but the benefit in accuracy more than compensates for the extra calculation time.  The equation to perform this calculation is shown in Eq.~\ref{eq:quad_approx}.

\begin{equation}
	\bar{\bm{\omega}} = \frac{1}{12}\left(-\bm{\omega}\left(t_{n-2}\right) + 8\bm{\omega}\left(t_{n-1}\right) + 5\bm{\omega}\left(t_n\right) \right)
	\label{eq:quad_approx}
\end{equation}

where $\bm{\omega}(t_{n-x})$ is the gyroscope measurement of the angular rate $x$ measurements previous.

The second modification is in the way that the attitude is propagated after finding $\bm{\dot{\hat{q}}}$.  Instead of performing the standar euler integration

\begin{equation}
	\bm{\hat{q}}_n = \bm{\hat{q}}_{n-1}+\bm{\dot{\hat{q}}}_{n}dt
\end{equation}

we use an approximation of the matrix exponential.  The matrix exponential arises out of the solution to the differential equation $\dot{x} = Ax$, namely
\begin{equation}
	x(t) = e^{At} x(0)
\end{equation}

and the discrete-time equivalent
\begin{equation}
	x(t_{n+1}) = e^{hA}(t_n)
\end{equation}

This discrete-time matrix exponential can be approximated by first expanding the matrix exponential into the infinite series

\begin{equation}
	e^{A} = \sum_{k=0}^\infty \dfrac{1}{k!}A^k
\end{equation}


and then grouping odd and even terms from the infinite series into two sinusoids, and results in the following equation for the propagation of the quaternion dynamics~\


\begin{equation}
	\qhat(t_n) = \left[ \cos \left( \frac{\norm{\w}h}{2}\right)I_4  + \frac{1}{\norm{\w}} \sin \left( \frac{\norm{\w}h}{2} \right) \lfloor\w\rfloor_4   \right] \qhat(t_{n-1})
	\label{eq:mat_exponential}
\end{equation}

where $\lfloor\w\rfloor_4$ is the 4x4 skew-symmetric matrix formed from $\w$

\begin{equation}
	\lfloor\w\rfloor_4 =
	\begin{bmatrix}
	    0	& - p & - q  & - r  \\
      	p   &   0   &  r & - q \\
      	q   & - r  &  0   &  p \\
      	r   &  q  & - p & 0  \\
	\end{bmatrix}
\end{equation}


\subsection{Tuning}
The filter can be tuned with the two gains $k_P$ and $k_I$.  Upon initialization, $k_P$ and $k_i$ are set very high, so as to quickly cause the filter to converge upon approprate values.  After a few seconds, they are both reduced by a factor of 10, to a value chosen through manual tuning.  A high $k_P$ will cause sensitivity to transient accelerometer errors, while a small $k_P$ will cause sensitivity to gyroscope drift.  A high $k_I$ will cause biases to wander unnecessarily, while a low $k_I$ will result in slow convergence upon accurate gyroscope bias estimates.  These parameters generally do not need to be modified from the default values.

\subsection{Implementation}
The entire filter is implemented in float-based quaternion calculations.  Even though the STM32F10x microprocessor does not contain a floating-point unit, the entire filter has been timed to take about 370$\upmu$s.  The extra steps of quadratic integration and matrix exponential propagation can be ommited for a 20$\upmu$s and 90$\upmu$s reduction in speed, respectively.  Even with these functions, however, this is sufficiently short to run at well over 1000Hz, which is the update rate of the MPU6050 on the naze32.
Control is performed according to euler angle estimates, and to reduce the computational load of converting from quaternion to euler angles (See Equation \ref{eq:euler_from_quat}), a lookup table approximation of atan2 and asin are used.  The Invensense MPU6050 has a 16-bit ADC and an accelerometer and gyro onboard.  The accelerometer, when scaled to $\pm$4g, has a resolution of 0.002394 m/s$^2$.  The lookup table method used to approximate atan2 and asin in the actual implementation is accurate to $\pm$ 0.001 rad.  Given the accuracy of the accelerometer, use of this lookup table implementation is justfied.  The C-code implementation of the estimator can be found in the file \begin{verbatim} src/estimator.c \end{verbatim}

\bibliographystyle{plain}
\bibliography{./library}


\end{document}
